\chapter{Introducción}
\label{chap:introduccion}
En el presente documento se muestra el resultado del \emph{Análisis Situacional del Trabajo} (AST) del sector del 
Cómputo Biomédico en el municipio de Cd. Victoria, Tamaulipas, cuyo propósito es identificar las funciones, tareas y operaciones de profesionales en el área del Cómputo Biomédico. Este análisis se basa en la información obtenida en el taller realizado en las instalaciones de la Universidad Politécnica de Victoria el día 8 de Septiembre del 2016.

Para la recolección de la información, se empleó la metodología para la elaboración de planes de estudios basados en competencias, adaptada en 1998 por el Ministerio de Educación en Quebec, así como la metodología DACUM (Developing a Curriculum), técnica desarrollada en 1960 en Canadá, como un medio rápido y efectivo para diseños curriculares, el cual ha sido utilizado ampliamente en Australia dentro de la educación superior para identificar las áreas de práctica, tareas y competencias en sus programas de estudio, así como en Estados Unidos a través del Centro de Educación y Formación para el empleo de la Universidad del Estado de Ohio.

Esta metodología se ha enriquecido por medio de la experiencia obtenida de su aplicación en diversos cotnextos y sectores del país, de tal forma que se pueden identificar, entre otros aspectos, todas aquellas habilidades psicomotoras, cognoscitivas y socio-afectivas, necesarias para ejercer una función productiva bajo criterios y estándares de desempeño.

El propósito de la aplicación de esta metodología es alinear los planes de estudio o programas de formación a los requerimientos y expectativas de los sectores estrechamente relacionados con la biomedicina del Estado de Tamaulipas y en su caso del país, lo que permitirá desarrollar los conocimientos, comportamientos, habilidades y destrezas acordes alas necesidades reales del entorno.

Para la obtención de la información, objetivo de esta análisis, se contó con el siguiente equipo de producción:

\begin{itemize}
	\item \textbf{Animador,} quien tuvo la responsabilidad central de generar la dinámica de la reunión y aplicar los mecanismos de recolección de información, así como de realizar el análisis de toda la información.
	
	\item \textbf{Especialistas del sector,} quienes proporcionaron la información necesaria para definir los requerimientos que se deben cumplir, a fin de satisfacer las necesidades del sector bajo estudio, como potenciales contratantes de este profesional.
	
	\item \textbf{Observadores, } quienes apoyaron en la recolección de toda la información.
\end{itemize}

Cabe hacer mención que el reporte generado a partir de este análisis, ha sido validado por las personalidades que participaron en el taller.


%%%%%%%%%%%%%%%%%%%%%%%%%%%%%%%%%%%%%%%%%%%%%%%%%%%%%%%%%%%%%%%%%%%%%%%%%%%%%%%%%%%%%%%%%%%%%%%%%%%%%%%%%%%%%%%%%%%%%%%%%%%%%%%%%%%%%
%%%		Section: Descripción General																						%%%%%%%%%
%%%%%%%%%%%%%%%%%%%%%%%%%%%%%%%%%%%%%%%%%%%%%%%%%%%%%%%%%%%%%%%%%%%%%%%%%%%%%%%%%%%%%%%%%%%%%%%%%%%%%%%%%%%%%%%%%%%%%%%%%%%%%%%%%%%%%
\section{Descripción General}
\label{sec:desc_general}

\subsection{Definición}
\label{subsec:definicion}

Además de las habilidades y conocimientos demandados a los profesionales que se desempeñan en el sector del Cómputo Biomédico, el perfil detectado en este Análisis Situacional del Trabajo tendrá entre sus funciones:

